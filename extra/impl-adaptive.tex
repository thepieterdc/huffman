\section{Adaptive Huffmancoding}
Voor de updateprocedure werd het algoritme uit de cursus \cite{ad3cursus}, pagina $102$ gebruikt. Wat in de cursus niet wordt besproken is hoe de knoop $t'$ op een effici\"ente manier kan worden gevonden. Knuth bespreekt in zijn paper \cite{knuthhuffman} een techniek waarbij dit mogelijk is in constante tijd, maar zelf bedacht ik een andere manier om deze knoop te bepalen. Mijn techniek heeft complexiteit $O(n)$, met $n$ het aantal knopen met exact hetzelfde gewicht als de knoop $t$. Na empirische tests blijkt de waarde van $n$ gemiddeld steeds onder $3$ te liggen, dus ook een zeer effici\"ente en vooral gemakkelijke manier. De sleutel van deze manier is eigenschap b van een Huffman boom, vermeld op pagina $96$ van de cursus \cite{ad3cursus}:
$$\textit{Als voor twee toppen }t, t'\textit{ geldt dat }o(t) < o(t')\textit{, dan geldt }a(t) \leq a(t')$$
Het algoritme voor het bepalen van deze knoop wordt beschreven in onderstaande pseudocode: