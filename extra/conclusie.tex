\chapter{Conclusie}

Na het afwegen van zowel compressieratio als tijd kunnen we besluiten dat het \huffstd in veel gevallen het beste presteert, op voorwaarde dat de data niet online moet worden ge\"encodeerd. Een belangrijk nadeel hiervan is een beperking op het aantal bytes dat ge\"encodeerd kan worden; de tekst moet namelijk in één keer in het geheugen passen. Een oplossing hiervoor zou kunnen zijn om de tekst in blokken op te delen en deze vervolgens via het \huffstd te encoderen. Indien de tekst online moet worden ge\"encodeerd, wordt bij voorkeur \huffadap gebruikt vanwege de snelheidswinst. Het \huffslid geeft de beste compressieratio maar neemt ook het meeste tijd in beslag, met gemiddeld een factor 15 ten opzichte van het \huffstd. Dit kan worden opgelost door middel van het \huffblock, mits een verlies aan compressieratio. In geen enkel geval is het \hufftwopass de optimale keuze. Wanneer de tekst voldoende \texttt{random} is, is de compressieratio voor elk algoritme gelijk. In dat geval spreekt het dus voor zich om ofwel het \huffstd te gebruiken, ofwel de tekst helemaal niet te encoderen. Als het bestand voldoende entropie bevat, kan het namelijk voorkomen dat de ge\"encodeerde tekst groter is dan de niet-ge\"encodeerde tekst.